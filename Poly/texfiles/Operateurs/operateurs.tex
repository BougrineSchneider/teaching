\begin{savequote}[80mm]
---Have you ever sat down and worked on a C++ project?  Here's
    what happens: First, I've put in enough pitfalls to make sure
    that only the most trivial projects will work first time. Take
    operator overloading.  At the  end of the project, almost every module
    has it, usually, because  guys feel they really should do it, as it
    was in their training  course.  The same operator then means something
    totally different in every module.
\qauthor{Parodie d'interview de Bjarne Stroustrup}
\end{savequote}
\chapter{Op\'erateurs}
\label{chapter:operateurs}

\section{Introduction}
Dans le chapitre \ref{chapter:classes}, nous avions \'ecrit un objet complexe, dont le code est rappel\'e ci-dessous :

\begin{lstlisting}[caption = complexe3.h]

class Complexe
{
    public:
        Complexe(double _rho, double _theta);
        ~Complexe();

        double module();
        double getReel();
        double setReel();
        double getImag();
        double setImag();
        double getArgument();

    private:
        double rho, theta;

}
\end{lstlisting}


Nous n'avions cependant pas \'ecrit de code permettant d'effectuer des op\'erations de calcul \'el\'ementaire sur ces complexes, par exemple une multiplication, ce que l'on peut faire assez
simplement. Il suffit simplement de rajouter une fonction au header,
\begin{lstlisting}[caption = complexe4.h]

class Complexe
{
    public:
        Complexe(double _rho, double _theta);
        ~Complexe();

        double module();
        double getReel();
        double setReel();
        double getImag();
        double setImag();
        double getArgument();
        Complexe multiplie(Complexe z);

    private:
        double rho, theta;

}
\end{lstlisting}

dont le code sera le suivant :
\begin{lstlisting}[caption = complexe3.cpp]
Complexe Complexe::multiplie(Complexe z)
{
    return Complexe(rho * z.rho, theta + z.theta);
}
\end{lstlisting}

Il s'agit simplement de renvoyer un nouveau nombre complexe avec les bons modules et arguments. D'un point de vue pratique, il est alors possible d'\'ecrire :
\begin{lstlisting}
double pi = 3.141592653;
Complexe z1(1, pi/4) , z2(2, pi/2), produit;

produit = z1.multiplie(z2);
\end{lstlisting}

Cependant, d'un point de vue lisibilit\'e, le code en question n'est pas tr\`es agr\'eable \`a manipuler. On pr\'ef\'ererait nettement le code suivant :
\begin{lstlisting}
double pi = 3.141592653;
Complexe z1(1, pi/4) , z2(2, pi/2), produit;

produit = z1 * z2;
\end{lstlisting}

En C++, les op\'erateurs sont des fonctions comme les autres, et en particulier, il est possible de les surcharger. Dans le cas de l'op\'erateur produit,
deux m\'ethodes sont possibles :
\begin{lstlisting}
//methode 1
Complexe operator*(Complexe & z)
// methode2
Complexe operator*(Complexe & z1, Complexe &z2);
\end{lstlisting}

Nous expliciterons les diff\'erences entre les deux d\'eclarations un peu plus loin, et allons utiliser pour les besoins de notre exemple la premi\`ere m\'ethode.
Nous obtenons donc le code suivant :
\begin{lstlisting}[caption = complexe5.h]
class Complexe
{
    public:
        Complexe(double _rho, double _theta);
        ~Complexe();

        double module();
        double getReel();
        double setReel();
        double getImag();
        double setImag();
        double getArgument();
        Complex operator*(Complexe & z);


    private:
        Complexe multiplie(Complexe z);
        double rho, theta;

}
\end{lstlisting}

et
\begin{lstlisting}[caption = complexe5.cpp]
Complexe Complexe::operator*(Complexe &z)
{
    return multiplie(Complexe z);
}
\end{lstlisting}

Quelques remarques :
\begin{itemize}
\item Il n'est pas obligatoire d'\'ecrire une fonction multiplie. JUSTIFIER POURQUOI ELLE EST LA
\item Cette derni\`ere a \'et\'e bascul\'ee en \texttt{private}, dans la mesure ou l'utilisateur n'a plus besoin de savoir qu'elle existe.
\end{itemize}

Muni de nouvel op\'erateur, le code suivant est d\'esormais parfaitement valable :
\begin{lstlisting}
double pi = 3.141592653;
Complexe z1(1, pi/4) , z2(2, pi/2), produit;

produit = z1 * z2;
\end{lstlisting}

Apr\`es cette br\`eve introduction, nous pouvons passer \`a une \'etude plus d\'etaill\'ee de la surcharge d'op\'erateurs.

\section{Etude d\'etaill\'ee}

\section{Que peut on surcharger?}

En C++, la quasi totalit\'e des op\'erateurs est surchargeable. La table
\ref{tab:operateurs} indique lesquels sont concern\'es.


\begin{table}
	\centering
	\begin{tabular}{c c}
	 1 & 2 \\
	\end{tabular}
	\caption{Table des op\'erateurs surchargeables}
	\label{tab:operateurs}
\end{table}


\section{Op\'erateurs internes, op\'erateurs externes}

\section{Quelques pr\'ecautions \`a prendre}
EXPLIQUER POURQUOI CERTAINS N'AIMENT PAS CA.


\begin{recapitulatif}
\item Il est possible de surcharger les op\'erateurs classiques, afin de permettre une manipulation plus ais\'ee \`a l'utilisateur de la classe.
\begin{lstlisting}
Complexe Complexe::operator+(Complexe &);
\end{lstlisting}

\item Un op\'erateur est soit externe, soit interne. DETAILLER.
\item Il faut prendre garde \`a la surcharge des op\'erateurs.
\end{recapitulatif} 