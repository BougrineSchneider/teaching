\chapter{Co�ts algorithmiques et containers}

\section{Rappels sur les notations de Landau}

Soit $\mathbb{E}$ l'ensemble des fonctions $g$ continues de $\mathbb{R}$ dans $\mathbb{R}$, telles que $g > 0$ au voisinage de $+\infty$. Soient $f$ et $g$ sont deux fonctions de $\mathbb{E}$.\\

Si la fonction $f/g$ est major�e par une constante au voisinage de $+\infty$, on �crit $f = O(g)$, que l'on lit "f est un grand o de g au voisinage de $+\infty$" , ou m�me par abus "f est un grand O de g".\\

Si la fonction $f/g$ tend vers 0 en $+\infty$, on �crit $f = o(g)$, que l'on lit "f est n�gligeable devant g au voisinage de $+\infty$" , "f est un petit o de g au voisinage de $+\infty$" ou m�me par abus "f est un petit o de g".\\

