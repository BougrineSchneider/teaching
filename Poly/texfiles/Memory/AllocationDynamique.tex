\chapter{Allocation dynamique }

Ce chapitre sera un peu plus th\'eorique que les autres, mais il expose des
notions indispensables \`a la bonne compr\'ehension du C++. Il est donc
\emph{important}\footnote{Il faut �galement le relire jusqu'\`a ce qu'il soit compris.}
et sera consacr\'e aux tableaux, \`a l'allocation dynamique de m\'emoire et aux
pointeurs.

\section{Tableaux}

Il est possible en C++ de d\'eclarer un \index{tableau}tableau de variables. Cela se fait de
la mani\`ere suivante :

\begin{lstlisting}[caption=D\'eclaration d'un tableau]
Type arrayName[size];
\end{lstlisting}

Par exemple,

\begin{lstlisting}
double someArray[ 50 ];
\end{lstlisting}
permet de d\'eclarer un tableau de 50 nombres d\'ecimaux en double pr\'ecision.\\

On acc\`ede alors aux \'el\'ements du tableau \`a l'aide de l'op\'erateur \texttt{[ ]}. Pour cr\'eer un tableau
de 10 \'el\'ements et y ranger les carr\'es des entiers de 1 \`a 10, on \'ecrirait:
\begin{lstlisting}[caption=Utilisation d'un tableau]
double someArray[ 10 ];

for( int i = 0 ; i < 10 ; i++)
{
    tableau[i]=(i+1)*(i+1); //Le tableau commence a l'indice 0
}
\end{lstlisting}

La syntaxe \'equivalente en Python ou en VBA serait :
\begin{center}%
    \begin{minipage}{0.45\linewidth}
       \begin{center}\emph{Python}\end{center}
\begin{lstlisting}[language=python]
#c'est une liste et
#non un tableau
tableau = []

for i in range(1, 11):
	tableau.append(i * i)
\end{lstlisting}
    \end{minipage}
    \qquad
    \begin{minipage}{0.45\linewidth}%
    \begin{center}\emph{VBA}\end{center}
\begin{lstlisting}[language=VBScript]
dim i as Integer
dim tableau(10) as Integer

for i=1 to 10
    tableau(i)=i*i
next i
\end{lstlisting}
    \end{minipage}
\end{center}

\section{Allocation dynamique }
Supposons \`a pr\'esent que nous souhaitions \'ecrire un programme qui demande \`a l'utilisateur de rentrer un nombre $n$.
Le programme en question demande ensuite \`a l'utilisateur de rentrer $n$ nombres successifs, pour ensuite
tous les r\'eafficher.
\subsection{Allocation dynamique}
Spontan\'ement, nous voudrions cr\'eer un tableau de taille $n$ au moyen du code suivant :

\includecodecaption{tableautaillevariableerreur.cpp}{Tableau de taille variable}

Malheureusement, ce code est erronn\'e. En effet, comme nous l'avons vu
pr\'ec\'edemment, chaque fois que l'on d\'eclare une variable (ou un tableau),
le compilateur C++ r\'eserve des cases m\'emoires. Mais pour les r\'eserver
correctement, il faut qu'il sache \emph{\`a la compilation}  combien de case il
doit r\'eserver, \emph{ce qui n'est pas le cas ici}. Il nous faut donc trouver une
autre solution.\\

Le probl\`eme se pose en ces termes : comment dire au compilateur " Je veux
r\'eserver n cases m\'emoires, \emph{pendant l'ex\'ecution et non pas � la compilation} du programme " ? Il
existe une fonctionalit�\footnote{Pour \^etre plus pr\'ecis, il s'agit en fait d'un op\'erateur d'un genre un peu particulier.} en C++ appel\'ee \texttt{new}, qui effectue pr\'ecis\'ement
cette op\'eration, et renvoie l'adresse (donc \emph{un pointeur}) du d\'ebut de
la zone m\'emoire ainsi r\'eserv\'ee. La syntaxe (dans notre cas) est la
suivante :

\begin{lstlisting}[caption=Exemple d' appel  \`a \texttt{new}]
/*La ligne suivante demande 10 entiers en memoire*/
int* someArray = NULL;
someArray = new int [10];
\end{lstlisting}

Si nous \'etudions cette ligne, nous pouvons faire deux remarques :

\begin{enumerate}
\item \texttt{new} renvoie un pointeur du type demand\'e.
\item \texttt{new} prend en param\`etre la taille du bloc \`a demander.
\end{enumerate}

\subsection{Arithm\'etique des pointeurs}

Une fois la m\'emoire r\'eserv\'ee, nous voulons pouvoir acc\'eder au bloc
m\'emoire en question. Le langage C++ permet de traiter la zone m\'emoire ainsi
r\'eserv\'ee exactement comme un tableau standard. Par exemple, le code suivant
r\'ealise ce que nous voulions faire au d\'ebut (rentrer des nombres, les
stocker, et les r\'eafficher) :

\includecode{tableautaillevariable.cpp}

Cependant, que se passe-t-il en r\'ealit\'e? Lorsque l'on \'ecrit
\begin{lstlisting}
someArray[i]=30;
\end{lstlisting}

c'est en fait l'instruction suivante qui est ex\'ecut\'ee :
\begin{lstlisting}
*(someArray + i)=30;
\end{lstlisting}

En d'autre termes, someArray contient bien l'adresse du d\'ebut du bloc
m\'emoire. En ajoutant i, on se d\'eplace \footnote{Attention! Il y a ici une
subtilit\'e. Saurez-vous la voir?} \`a l'adresse du i-\`eme \'el\'ement, dont
on modifie la valeur \`a l'aide de l'op\'erateur de d�r�f�rentiation \texttt{*}. Le sch\'ema
ci-dessous repr\'esente la situation en m\'emoire lorsque l'on a un tableau de
taille 5 :

\begin{center}
\begin{tabular}{l|l|l|l|l|l|l|l|l|l|l}
Case m\'emoire & 0 & 1 & 2 & 3 & 4 & 5 & 6 & 7 & 8 & 9 \\
\hline
Nom &  &  & someArray &  & \multicolumn{5}{c|}{m\'emoire r\'eserv\'ee pour le tableau} & \multicolumn{1}{l|}{} \\
\hline
Valeur &  &  & case m�moire 4 &  &  &  &  &  &  &  \\
\end{tabular}
\end{center}

Etudions le d\'eroulement du code suivant :
\begin{lstlisting}
int* tableau;

/*etape 1*/
tableau = new int [ 5 ];

/*etape 2*/
tableau[2] = 37;

/*etape 3*/
*(tableau + 2) = 42;
\end{lstlisting}

Apr\`es l'\'etape 1, nous sommes dans la situation ci-dessous :

\begin{center}
\begin{tabular}{l|l|l|l|l|l|l|l|l|l|l}
Case m\'emoire & 0 & 1 & 2 & 3 & 4 & 5 & 6 & 7 & 8 & 9 \\
\hline
Nom &  &  & tableau &  & \multicolumn{5}{c|}{m\'emoire r\'eserv\'ee pour tableau} & \multicolumn{1}{l|}{} \\
\hline
Valeur &  &  & Case m�moire 4 &  &  &  &  &  &  &  \\
\end{tabular}
\end{center}

La m\'emoire r\'eserv\'ee n'est \emph{pas} initialis\'ee \`a 0 (ou tout autre valeur).

Apr\`es l'\'etape 2,
\begin{center}
\begin{tabular}{l|l|l|l|l|l|l|l|l|l|l}
Case m\'emoire & 0 & 1 & 2 & 3 & 4 & 5 & 6 & 7 & 8 & 9 \\
\hline
Nom &  &  & someArray &  & \multicolumn{5}{c|}{m\'emoire r\'eserv\'ee pour tableau} & \multicolumn{1}{l|}{} \\
\hline
Valeur &  &  & Case m�moire 4 &  &  &  & 37  &  &  &  \\
\end{tabular}
\end{center}

le troisi\`eme \'el\'ement du tableau a \'et\'e modifi\'e.

Apr\`es l'\'etape 3,
\begin{center}
\begin{tabular}{l|l|l|l|l|l|l|l|l|l|l}
Case m\'emoire & 0 & 1 & 2 & 3 & 4 & 5 & 6 & 7 & 8 & 9 \\
\hline
Nom &  &  & someArray &  & \multicolumn{5}{c|}{m\'emoire r\'eserv\'ee pour tableau} & \multicolumn{1}{l|}{} \\
\hline
Valeur &  &  & Case m�moire 4 &  &  &  &  42 &  &  &  \\
\end{tabular}
\end{center}

le m\^eme \'el\'ement a \'et\'e modifi\'e, mais en \'ecrivant explicitement o\`u il se trouvait en m\'emoire.


Il est int\'eressant de remarquer que \texttt{*tableau = *(tableau + 0) =
tableau[0]} correspond au premier \'el\'ement du tableau. C'est une des raisons pour
lesquelles les tableaux en C++ commencent \`a l'indice 0 et non 1.\\


Une subtilit\'e s'est cependant gliss\'ee ici, qui a \'et\'e pass\'ee sous
silence pour des raisons de simplicit\'e. En effet, suivant son type, une
variable n'occupe pas le m\^eme nombre de cases m\'emoires : un \texttt{int} va
g\'en\'eralement\footnote{Le cas des \texttt{int} est un peu particulier,
puisqu'il d\'epend du type de processeur/compilateur utilis\'e. Sur les processeurs 32 bits,
ce sera 32 bits soit 4 octets (cases).} occuper 4 cases, un \texttt{char}
1 case, et un \texttt{double} 8 cases :

\begin{center}
\begin{tabular}{l|l|l|l|l|l|l|l|l|l|l|l|l|l}
Case m\'emoire & 0 & 1 & 2 & 3 & 4 & 5 & 6 & 7 & 8 & 9 & 10 & 11 & 12 \\
\hline
Nom & \multicolumn{1}{c|}{char c} & \multicolumn{4}{c|}{int n} & \multicolumn{8}{c|}{double d} \\
\hline
Valeur &  &  &  &  &  &  &  &  &  &  &  &  &  \\
\end{tabular}
\end{center}

Pourquoi, lorsque l'on \'ecrit
\begin{lstlisting}
*(tableau + 2) = 42;
\end{lstlisting}

se trouve-t-on \`a l'adresse \texttt{tableau + 2* sizeof(int)\footnote{\functionname{sizeof} est une fonction particuli\`ere qui renvoie la taille en m\'emoire (le nombre de cases) d'une variable.}}  et non
\texttt{tableau + 2} (c'est � dire au milieu d'un entier)? Pour des raisons de confort,
lorsque l'on manipule des pointeurs, le compilateur C++ ajoute automatiquement le
\texttt{sizeof} n\'ecessaire. Cette arithm\'etique un peu particuli\`ere -on
ajoute 2 \emph{en apparence} - porte le nom d'arithm\'etique des pointeurs.

\subsection{Nettoyage de la m\'emoire}

Nous savons \`a pr\'esent allouer de la m\'emoire selon nos besoins, en cours
d'ex\'ecution du programme. Qu'advient-il de cette m\'emoire allou\'ee  lorsque notre
code ne l'utilise plus? La r\'eponse est simple : elle reste allou\'ee. Pour
cela, nous devons comprendre ce qui se passe lorsque nous allouons de la
m\'emoire.\\

L'OS (syst\`eme d'exploitation) d'un ordinateur
est un "programme" qui tourne en permanence, et qui se charge (entre autres) de
faire cohabiter les diff\'erents programmes tournant simultan\'ement les uns
avec les autres. En particulier, lui seul dispose d'un contr\^ole total sur la
m\'emoire de l'ordinateur. En pratique, chaque fois qu'un programme  demande de
la m\'emoire au moyen d'une instruction \texttt{new} ou similaire, c'est en
fait au syst\`eme d'exploitation qu'il la demande. Celui ci r\'eserve alors une
zone, et la marque comme appartenant au programme demandant la m\'emoire.\\

Pendant l'ex\'ecution du programme, l'OS n'a \emph{aucune} raison de d\'esallouer la
m\'emoire qui a \'et\'e demand\'ee, et ce quelle que soit la situation. On
pourrait tr\`es bien imaginer des cas dans lesquels un utilisateur a lanc\'e
une trentaine de programme simultan\'ement qui consomment tous beaucoup de
m\'emoire, et o\`u l'OS aurait int\'er\^et, voire besoin de lib\'erer de la
m\'emoire. Cependant, il ne le fera jamais. Certains langages (Java, Python,
Pascal, VBA, etc.) fournissent un  m\'ecanisme de nettoyage automatique
\footnote{appel\'e ramasse-miette, ou \textit{garbage collection}} de la m�moire qui n'est plus utilis�e,
mais ce n'est pas le cas du C++ \footnote{Au moins � notre niveau}.\\

Afin d'\'eviter ce genre de situation, la m�moire allou�e dans un programme doit \emph{toujours} �tre lib\'er�e d�s qu'elle n'est plus utilis�e. Cela se fait
au moyen de l'instruction \texttt{delete}. Cette instruction prend deux formes:

\begin{itemize}
	\item Si la m�moire allou�e �tait destin�e � un tableau, elle doit �tre d�sallou�e avec l'op�rateur delete[] :
\begin{lstlisting}
delete[] someArray;
\end{lstlisting}
	\item Si la m�moire allou�e �tait destin�e � une variable, on utilisera une forme sans crochet:
\begin{lstlisting}
delete pSomeElement;
\end{lstlisting}
\end{itemize}

\includecodecaption{newdelete.cpp}{Tableau de taille variable avec nettoyage de la m\'emoire}

\begin{habitudes}[\texttt{new}/\texttt{delete}]
	
\`A chaque fois que l'on \'ecrit un \functionname{new}, il faut \'ecrire imm�diatement le
\functionname{delete} associ\'e. Cela permet d'\^etre certain de ne pas
l'oublier, et donc d'\'eviter bien des probl\`emes plus tard.

\end{habitudes}

\section{Quelques pr\'ecautions}

Ce chapitre est sans doute le plus difficile \`a comprendre mais le plus
important de ce cours : une fois compris, l'ensemble des notions abord\'ees par
la suite para\^itront simple. Le langage C++ est un langage qui peut-\^etre consid\'er\'e comme de \emph{bas-niveau}
-par opposition \`a des langages comme Java, Python, Pascal, VBA, etc., qui
sont dits de haut niveau- dans la mesure o\`u il ne dissimule pas au
programmeur ce qui se passe dans l'ordinateur. Cela a des avantages (contr\^ole
tr\`es fin sur le d\'eroulement du programme, optimisations plus faciles \`a
effectuer pour le compilateur, meilleures performances \footnote{Ceci est de plus en plus discutable, cf les derniers benchmarks http://shootout.alioth.debian.org/, il faut �tre tr�s prudent lorsque l'on parle de performances}), et des inconv\'enients
(obligation de penser "comme la machine", plus grande difficult\'e \`a \'ecrire
du code robuste, augmentation du temps de d\'eveloppement)

Il est important de noter que plus de 50\% des bogues rencontr\'es lors du d\'eveloppement d'application sont dus \`a :
\begin{itemize}
\item des d\'epassements d'indices dans des tableaux
\item de la m\'emoire lib\'er\'ee plusieurs fois
\item des pointeurs non initialis\'es ou pointant sur des zones m\'emoires non-initialis\'ees.
\end{itemize}
Il convient donc, lorsque l'on \'ecrit du code manipulant de la m\'emoire, de le faire avec la plus grande prudence.\\

R�capitulatif :

\begin{itemize}
\item Un tableau est d\'eclar\'e en C++ de la mani\`ere suivante:
\begin{lstlisting}
typeDeVariable nomTableau[nbElements];
\end{lstlisting}
\item Si l'on a besoin d'un tableau de taille variable, ou inconnue \`a la compilation, il faut allouer dynamiquement de la m\'emoire, c'est-\`a-dire la demander
au syst\`eme au moyen de l'op�rateur \functionname{new}. Celle-ci renvoie un pointeur sur une zone m\'emoire, qui peut \^etre manipul\'ee comme un tableau.
Par exemple,
\begin{lstlisting}
int* tableau = new int[40];
tableau[27] = 42;
\end{lstlisting}
\item Tout objet (tableau ou non) construit via l'op�rateur new doit �tre d�truit manuellement via l'op�rateur delete (ou delete[] dans le cas d'un tableau).
\end{itemize} 